\documentclass[a4paper, 11pt]{article}
\usepackage[czech,slovak,english]{babel}
\usepackage[utf8]{inputenc}
\usepackage{url}
\usepackage{fullpage}
\usepackage{hyperref}
\def\UrlBreaks{\do\/\do-}
\title{Model domovov pre dôchodcov}

\begin{document}
\begin{center}
\Large \textbf{Model domovov pre dôchodcov}
\end{center}
\noindent
\large\textbf{Modelovanie domova pre seniorov (5)} \hfill \textbf{Ondrej Kurák}, xkurak00 \\
\today \hfill \textbf{Martin Bažík}, xbazik00 \\


\section{Úvod}
V tejto práci je riešená implentácia modelu\cite[str. 7]{IMS} napĺňania kapacít domovov dôchocov a s tým spojená implentácia modelu vývoja demografie obyvateľstva. Na základe vytvoreného modelu a simulačných experimentov\cite[str. 8]{IMS} sa pokúsime vytvoriť predpoklad pre potrebné kapacity domovov dôchodcov v juhomaravskom kraji. Jednotlivé experimenty sa zameriavajú na odhady vývoja aktuálneho stavu zaplnenia domovov dôchodcov.
\subsection{Autori}
\begin{itemize}
\item Martin Bažík (xbazik00)
\item Ondrej Kurák (xkurak00)
\end{itemize}
\subsection{Odborný konzultant}
\begin{itemize}
\item MUDr. Milan Kurák
\end{itemize}
\subsection{Zdroje faktov}
Ako zdoje faktov boli použité správy z Ústavu zdravotných informácii a štatistýk Českej republiky\cite{demografia}\cite{domovy}, správy Českého štatistického úradu\cite{zomreli}. 
\subsection{Validita modelu}
Validitu modelu\cite[str. 37]{IMS} pre napĺňanie kapacít domovov dôchodcov nie je možné úplne overiť.

Validita modelu pre vývoj demografie obyvatelstva je čiastočne overená na základe porovnávania reálnej hrubej miery úmrtnosti v minulosti a údajov zistených z experimentov. Kvôli dĺžke experimentu sa presnosť modelu stále znižuje.

\section{Rozbor témy a použitých metód/technológií}
Práca sa zaoberá modelom napĺňania kapacít domovov pre dôchodcov a vývoj demografie v juhomoravskom kraji. 

Domov pre dôchodcov je zariadenie, do ktorého prichádzajú klienti po dosiahnutí určitého veku. Pre tento projekt sme zvolili vek 65 rokov, ktorý je možné odvodiť na základe tabuľky \texttt{Důchodový věk}\cite{duchod} pri súčasnej dendencii navyšovania dôchodkového veku. Priemerný vek mužov, ktorý nastúpia do dôchodku od roku 2017 do roku 2044, je 65 rokov.

Hlavným faktorom napĺňania domova dôchodcov je starnutie populácie. Stredná dĺžka života na základe tabuľky \texttt{Vývoj ukazatelů úmrtnosti} \cite[str. 8]{zomreli} rastie, pričom v poslednom meranom roku 2016 dosiahol medziročný nárast o 0,5\%.

Okrem starnutia populácie má vplyv na napĺňanie domovov pre dôchodcov aj samotná kapacita domovov pre dôchodcov. Pre tento projekt sme zvolili konštatnú kapacitu, ktorá je pre juhomoravský kraj definovaná na základe tabuľky\cite{miesta}. Súčasný stav tvorí 4128 miest.  

Pre modelovanie demografie populácie sú potrebné dáta o súčasnom stave populácie\cite{demografia} a obsadení domovov pre dôchodcov\cite{domovy}, úmrtnosť jednotlivých vekových kategórii\cite{zomreli}, ako aj pôrodnosť. Nakoľko model obmedzujeme do roku 2050, je pôrodnosť zanedbateľná. Novonarodení jedinci totiž nemôžu dosiahnuť dôchodkový vek vo vymedzenom časovom období.

Jednotlivé dáta sú agregované v päťročných kategóriách a pre tento projekt ich bolo potrebné spracovať pre jednotlivé roky.

%doplnit, neviem ake fakty este vyuzivame
%odchod do domova

\subsection{Použité postupy}
%treba zdôvodniť prečo Python.
Pre simulačný model je využitá diskrétna simulácia\cite[str. 34]{IMS}. Tento model je vytvorený v programovacom jazyku Python3. Dôvodom pre využitie programovacieho jazyka Python3 je možnosť vyuitia vedeckých knižníc Numpy a Scipy. Tieto knižnice obsahujú matematické metódy, ktoré zjednodušujú spracovanie vstupných dát a poskytujú možnosť aproximovať krivky. Ak alternatívu by bolo možné použiť knižnicu GNU Scientific Library jazyka C, ktorá tiež poskytuje vedecké matematické metódy. %zdôvodnit preco nie, vizualizacia?
\subsection{Použité technológie}

%treba vypisat co vsetko tam bolo pouzite a nejaky odkaz


\section{Koncepcia modelu}
\subsection{Spôsob vyjadrenia modelu}
\subsection{Popis konceptuálneho modelu}


\section{Architektúra simulačného modelu/simulátoru}
Hlavnou zložkou modelu je predpoveď demografie obyvatelstva.

\subsection{Použité postupy}
Všetky dáta, ktoré je možné pre demografiu obyvatelstva a domovy dôchodcov sú agregované do kategórii po piatich rokoch. Aby bolo možné modelovať jednotlivé zložky je potrbené tieto údaje vhodne rozložiť.

\subsubsection*{Predpovedanie percentuálnej úmrtnosti obyvateľstva}
K úmrtnosti obyvatelsta disponujeme dátamy o úmrtnosti v percentách pre jednotlivé vekové kategórie (agregovaných po piatich rokoch) od roku 2005.

Pre prepovedanie percentuálne pravdepodonosťi úmrtnosti vekových kategórii aproximujeme nelineárnu funkciu na akuálne dáta pre každú vekovú kategóriu. Funkcia je v tvare:
$$f(x)=\frac{a_0}{x^{a_1} + a_2}$$
Aproximácia prebieha zmenou koeficientov $a_0, a_1, a_2$ metódou najmenších štvorcov. Daný typ funkcie bol zvolený, pretože dáta majú zostupný charakter a zároveň predpokladáme, že úrtnosť nemôže dosiahnuť 0\%.

Aby bolo možné určiť percentuálnu úmrtnosť pre každú vekovú kategóriu (nie kategórie spojené po piatich rokoch, vekové kategórie nad 85 rokov) pomocou predošlých funkcii pre každý sledovaný rok vygenerujeme body pre každú agregovaná vekovú kategóriu. Následne aproximujeme nelineárnu funkciu cez takto vygenerované body. Funkcia je v tvare:
$$f(x) = a_0 * x^a_1 + a_2$$
Aproximácia prebieha zmenou koeficientov $a_0, a_1, a_2$ metódou najmenších štvorcov. Daný typ funkcie bol zvolený, pretože dáta majú vzostupný charakter a konvexný tvar.

Pomocou takto pripravených funkcii následne dokážeme prepovedať percentálnu úmrtnosť každej vekovej kategórie.

Presnosť predpovede klesá každý nasledujúci rokom.

\subsubsection*{Počiatočné rozloženie obyvateľstva}
Kedže počiatočný stav jednotlivých vekových kategórii obyvatelstva je tak isto agregovaný po piatich rokoch robíme jeho rozloženie na 5 častí. Aby rozloženie čo najviac zodpovedalo realite, vychádzame z niekoľkých prepokladov:
\begin{enumerate}
\item Počet narodených detí sa päťročné obdobie mení minimálne.
\item Počet ľudí v nasledujúcej vekovej kategórii by mal byť menší o počet ľudí, ktorí umreli v poslednom roku.
\end{enumerate}
Tieto predpoklady boli vyvodené na základe pozorovania týchto veličín na dlhšom období. Zároveň minimálny vek, ktorý sa pri strope v roku 2045 môže dostať do dôchodkového veku je 36. Od tohto veku je počet ľudí vo vekových kategóriach skoro klesajúca postupnosť.

Počet ľudí pre agregovanú vekovú kategóriu vyjadriť ako $y = x_{k} + x_{k+1} + x_{k+2} + x_{k+3} + x_{k+4}$, kde $x_0$ až $x_4$ sú počty ľudí jednotlivých vekových kategórii. Za pomoci predpokladov a percentuálnej úmrtnosti pre jendotlivé vekové kategórie vieme vyjadriť v tvare $x_{k+1}=x_{k}*P(k)$ kde $P(k)$ je doplnok k pravdepodobnosti úmrtnosti danej vekovej kategórie. Následne dokážeme vyjadriť počet ľudí pre agregované kategórie ako:
$$y = x_k*(1 + P(k)*(1 + P(k+1)*(1 + P(k+2) * (1 + P(k+3)))))$$
Týmto spôsobom vieme rozdeliť obyvatelstvo tak, aby sme zachovali počet ľudí pre agregovanú kategóriu a zároveň ich rozdelili s dôrazom na úmrtnosť.
\subsection{Použité technológie}


\section{Podstata simulačných experimentov a ich priebeh}

\section{Zhrnutie simulačných experimentov a záver}

\renewcommand\refname{Odkazy}
\begin{thebibliography}{9}
\bibitem{IMS} Peringer,P.: Modelování a simulace. FIT VUT, [ cit. 29. novembra 2017 ]. Dostupné na: \\\url{https://www.fit.vutbr.cz/study/courses/IMS/public/prednasky/IMS.pdf}

\bibitem{demografia} Kolektív autorov: Počty obyvatel v okresech ČR dle 5ti letých věkových kategorií. Opendata.cz, [ cit. 27. novembra 2017 ]. Dostupné na: \\\url{https://linked.opendata.cz/dataset/czso-demography-in-regions-czech-republic-age-categories}

\bibitem{domovy} Kolektív autorov: Zařízení sociálních služeb a domy s pečovatelskou službou v okresech ČR. Opendata.cz, [ cit. 27. novembra 2017 ]. Dostupné na: \\\url{https://linked.opendata.cz/dataset/czso-social-service-facilities}

\bibitem{JM_demografia} Kolektív autorov: Juhomoravský kraj. Wikipédia, [ cit. 27. novembra 2017 ]. Dostupné na: \\\url{https://sk.wikipedia.org/wiki/Juhomoravsk%C3%BD_kraj}

\bibitem{zomreli} Kolektív autorov: Zemřelí 2016. ÚZIS, 2017, [ cit. 27. novembra 2017 ]. Dostupné na:
\\\url{http://uzis.cz/publikace/zemreli-2016}

\bibitem{populacia} Kolektív autorov: Obyvatelstvo Česka. Wikipédia, [ cit. 27. novembra 2017 ]. Dostupné na:\\\url{https://cs.wikipedia.org/wiki/Obyvatelstvo_%C4%8Ceska#Popula.C4.8Dn.C3.AD_statistiky}

\bibitem{miesta} WWW stránka: Domovy pro seniory. GERONTOLOGIE, [ cit. 27. novembra 2017 ]. Dostupné na: 
\url{http://www.gerontologie.cz/showdoc.do?docid=30&typzar=DDP&kraj=Jihomoravsk%FD}

\bibitem{duchod} WWW stránka: Starobní důchody. ČSSZ, [ cit. 29. novembra 2017 ]. Dostupné na: 
\url{http://www.cssz.cz/cz/duchodove-pojisteni/davky/starobni-duchody.htm}


\end{thebibliography}

\end{document}
