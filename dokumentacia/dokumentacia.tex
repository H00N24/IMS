\documentclass[a4paper, 11pt]{article}
    \usepackage[czech,slovak,english]{babel}
    \usepackage[utf8]{inputenc}
    \usepackage{url}
    \usepackage{fullpage}
    \usepackage{hyperref}
    \def\UrlBreaks{\do\/\do-}
    
    \begin{document}
    \begin{center}
    \Large \textbf{Model domovov pre dôchodcov}
    \end{center}
    \noindent
    \large\textbf{Modelovanie domova pre seniorov (5)} \hfill \textbf{Ondrej Kurák}, xkurak00 \\
    \today \hfill \textbf{Martin Bažík}, xbazik00 \\
    
    
    \section{Úvod}
    V tejto práci je riešená implentácia modelu naĺňania kapacít domovov dôchocov a s tým spojená implentácia modelu vývoja demografie obyvateľstva. Na základe vytvoreného modelu a simulačných experimentov sa pokúsime vytvoriť predpoklad pre potrebné kapacity domovov dôchodcov v juhomaravskom kraji. Jednotlivé experimenty sa zameriavajú na odhady vývoja aktuálneho stavu zaplnenia domovov dôchodcov.
    \subsection{Autori}
    \begin{itemize}
    \item Martin Bažík (xbazik00)
    \item Ondrej Kurák (xkurak00)
    \end{itemize}
    \subsection{Odborný konzultant}
    \begin{itemize}
    \item MUDr. Milan Kurák
    \end{itemize}
    \section{Zdroje faktov}
    Ako zdoje faktov boli použité správy z Ústavu zdravotných informácii a štatistýk Českej republiky, správy Českého štatistického úradu. 
    \subsection{Validita modelu}
    Validita modelu pre napĺňanie kapacít domovov dôchodcov nie je možné úplne overiť.
    
    Validita modelu pre vývoj demografie obyvatelstva je čiastočne overená na základe porovnávania reálnej hrubej miery úrtnosti v minulosti a údajov zistených z experimntov. Kvôli dĺžke experimentu sa presnosť modelu stále znižuje.
    
    \section{Rozbor témy a použitých metód/technológií}
    Práca sa zaoberá modelom napĺňania kapacít domovov pre dôchodcov a vývoj demografie v juhomoravskom kraji.
    \subsection{Použité postupy}
    \subsection{Použité technológie}
    
    
    \section{Koncepcia modelu}
    \subsection{Spôsob vyjadrenia modelu}
    \subsection{Popis konceptuálneho modelu}
    
    
    \section{Architektúra simulačného modelu/simulátoru}
    Implementácia modelu prebehla v jazyku Python a knižníc numpy, scipy, matplotlib. Tieto technológie boli zvolené, kvôli potrebe vytvárania predpokladov na percentuálnu úmrtnosť jednotlivých skupín obyvatelstva.
    \subsection{Použité postupy}
    \subsection{Použité technológie}
    
    
    \section{Podstata simulačných experimentov a ich priebeh}
    
    \section{Zhrnutie simulačných experimentov a záver}
    
    \renewcommand\refname{Odkazy}
    \begin{thebibliography}{9}
    \bibitem{demografia} Kolektív autorov: Počty obyvatel v okresech ČR dle 5ti letých věkových kategorií. Opendata.cz, [ cit. 27. novembra 2017 ]. Dostupné na: \\\url{https://linked.opendata.cz/dataset/czso-demography-in-regions-czech-republic-age-categories}
    
    \bibitem{domovy} Kolektív autorov: Zařízení sociálních služeb a domy s pečovatelskou službou v okresech ČR. Opendata.cz, [ cit. 27. novembra 2017 ]. Dostupné na: \\\url{https://linked.opendata.cz/dataset/czso-social-service-facilities}
    
    \bibitem{JM_demografia} Kolektív autorov: Juhomoravský kraj. Wikipédia, [ cit. 27. novembra 2017 ]. Dostupné na: \\\url{https://sk.wikipedia.org/wiki/Juhomoravsk%C3%BD_kraj}
    
    \bibitem{zomreli} Kolektív autorov: Zemřelí 2016. ÚZIS, 2017, [ cit. 27. novembra 2017 ]. Dostupné na:
    \\\url{http://uzis.cz/publikace/zemreli-2016}
    
    \bibitem{populacia} Kolektív autorov: Obyvatelstvo Česka. Wikipédia, [ cit. 27. novembra 2017 ]. Dostupné na:\\\url{https://cs.wikipedia.org/wiki/Obyvatelstvo_%C4%8Ceska#Popula.C4.8Dn.C3.AD_statistiky}
    
    \bibitem{miesta} WWW stránka: Domovy pro seniory. GERONTOLOGIE, [ cit. 27. novembra 2017 ]. Dostupné na: 
    \url{http://www.gerontologie.cz/showdoc.do?docid=30&typzar=DDP&kraj=Jihomoravsk%FD}
    
    \end{thebibliography}
    
    \end{document}
    